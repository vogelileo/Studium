\documentclass{article}
\usepackage{amsmath}
\title{Taylor Reihen}
\author{Leo Vogel}
\begin{document}
\maketitle
\section*{Taylor Reihen}
\underline{Motivierendes Beispiel:} Berechne \(\ln(2)\) möglichst genau
\newline
Skizze:

Gemäss Skizze \(t(2) \approx \ln(2)\)

\underline{Formel für die Tangente:} \(f(x) = \ln(x)\)
\[
\begin{cases}
    t(x_0) = f(x_0) \\
    t'(x_0) = f'(x_0)
\end{cases}
\]

\(
\Rightarrow t(x) = f''(x)(x - x_0) + f(x_0)
\)

\underline{Approximation von \(\ln(2)\)} \(x_0 = 1\) 

\(f(x)=\ln(x)\); \(f(x_0)=\ln(x_0)=\ln(1)=0\)

\(f'(x) = \frac{1}{x}\); \(f'(x_0)=\frac{1}{x_0}=\frac{1}{1}=1\)

\(
t(x) = 1 \cdot (x-1)+0=x-1
\)

\(
\ln(2) \approx t(2) = 2-1=1
\)
\(
\ln(2) \approx (1) 
\) 
 


\end{document}